\documentclass[a4paper,11pt]{article}
\usepackage{latexsym}
\usepackage[polish]{babel}
\usepackage[utf8]{inputenc} 
\usepackage[MeX]{polski}
\usepackage{nicefrac}
\usepackage{listings}

\author{Marcin Krzyżaniak  \\ Maciej Spychała}

\title{Przetwarzanie i rozpoznawanie obrazu\\ 
\large{{\bf Sprawozdanie} \\ Projekt 1}} 

\begin{document}

\maketitle 

\section{Przetwarzanie obrazu}
\begin{enumerate}
\item Obraz wejsciowy poddajemy dylatacji.
\item Od obrazu poddanego dylatacji odejmujemy wejsciowy obraz. W ten sposób otrzymujemy sam kontur obrazu.
\item Za pomocą funkcji ConvexHull z biblioteki scipy znajdujemy otoczkę wypukłą konturu.
\item Dla każdego punktu w otoczce obliczamy kąt jaki tworzą wychodzące z niego odcinki. Jeżeli kąt jest większy niż 65 i mniejszy niż 115, wtedy traktujemy go jako potencjalny wierzchołek przy podstawie naszej figury.
\item Punkty na otoczce z odpowiednim kątem grupujemy wg. odległości od siebie, tzn. tworzymy grupy punktów leżących blisko siebie.
\item Dla każdej grupy punktów prowadzimy prostą od pierwszego do ostatniego punktu i jako potencjalny wierzchołek prostokąta traktujemy punkt z grupy o największej odległości od tej prostej.
\item Dla ''najlepszych'' punktów z każdej grupy wybieramy te 2, które są połączone odcinkiem i mają największą odległość od siebie. To są wierzchołki wejściowego prostokąta.
\item Dla obu wierzchołków szukamy punktu z otoczki który jest połączony z wierzchołkiem odcinkiem i posiada największą odległość od niego, ale nie leży na odcinku łączącym wierzchołki. Otrzymane punkty traktujemy jako miejsce przecięcia.
\item Mając 4 punkty, znajdujemy przekształcenie perspektywiczne metodą getPerspectiveTransform z biblioteki OpenCV, a następnie odwracamy je tak aby krawędź prostokąta była dolną ktawędzią obrazu po przekształceniu, a punkty przecięcia byly na bocznych krawędziach obrazu w proporcjach takich jak przed przekształceniem.
\item Na podstawie otrzymanego obrazu tworzymy wektor wartości na krawędziach. Każda kolejna liczba w wektorze to suma zapalonych pikseli w kolejnej kolumnie. Następnie od każdej wartości odejmujemy najmniejszą wartość z otrzymanego wektora, a następnie normalizujemy aby wartości należały do przedziału  (0,249).
\item Tak otrzymany wektor można interpretować jako znormalizowaną krzywą przecięcia prostokąta, czyli to co w dalszej części programu użyjemy do rozpoznawania obrazu.
\end{enumerate}
\subsection{Mierzenie kątów}
Aby zmierzyć kąt jaki tworzą odcinki wychodzące z wierzchłka, skanujemy miejsca przecięcia kwadratu, , którego środek jest badanym wierzchołkiem, z tymi odcinkami, a następnie wyznaczamy kąt (przecięcie\_1, wierzchołek, przecięcie\_2). Jeżeli w naszym kwadracie występują więcej niż 2 przecięcia uznajemy, że dany wierzchłek na pewno leży na krzywej, a więc nie jest wierzchołkiem prostokąta.

\section{Rozpoznawanie obrazu}
Dla każdego wektora (obrazka) obliczamy sumę wartości bezwzględnych rożnic kolejnych kolumn z każdym innym wektorem. Z racji tego, że nie wiemy czy nasz obraz nie został przypadkiem odbity lustrzanie podczas transformacji, dla każdego wektora wybieramy najlepszą (najmniejszą) sumę z 4 przekształceń (normalny obraz, odbicie w poziomi, odbicie w pionie, odbicie w pionie i poziomie). Następnie tworzymy ranking dopasowań obrazków, na podstawie tych sum (najmniejsza suma to najmniejsza rożnica krzywych, a więc najlepsze dopasowanie). Aby nie dopasowywać niepewnych obrazków do tych, których parę znaleźliśmy prawdopodobnie perfekcyjnie, wprowadziliśmy próg sumy, poniżej którego uznajemy nasze rozpozanie za pewne. Następnie usuwamy z rankingów obrazków niepewnych, wszystkie obrazki z początku rankingu, które wskazałiśmy jako pewne.
\bigskip
\noindent


\end{document}  
